%Este trabalho está licenciado sob a Licença Creative Commons Atribuição-CompartilhaIgual 3.0 Não Adaptada. Para ver uma cópia desta licença, visite http://creativecommons.org/licenses/by-sa/3.0/ ou envie uma carta para Creative Commons, PO Box 1866, Mountain View, CA 94042, USA.
%!TEX root = ../livro.tex

\setlength{\headheight}{30pt} % Veja https://nw360.blogspot.com/2006/11/latex-headheight-is-too-small.html
\chapter{Cálculo}\label{cap_introducao}
O Cálculo é muito importante para ganhar 6 créditos na engenharia.

\section{Reais: um conjunto arquimediano}\label{sec_def_trans_lap}

\begin{defn}
    Um número é dito positivo se for maior que zero.
\end{defn}

1

\subsection*{Exercícios}
~
\begin{exer} Calcule $2 + 3\times 4 + 1$.
\end{exer}
\begin{resp}
15
\end{resp}

\begin{exer} Calcule:
    \begin{equation}
        1 + 2 + 3 + 4 + 5 + \cdots + 100
    \end{equation}
\end{exer}
\begin{resp}
    5050
\end{resp}
    
